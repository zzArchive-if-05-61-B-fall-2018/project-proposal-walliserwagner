\documentclass[12pt]{article}
\usepackage{geometry}                % See geometry.pdf to learn the layout options. There are lots.
\geometry{letterpaper}                   % ... or a4paper or a5paper or ... 
\usepackage{graphicx}
\usepackage{amssymb}
\usepackage{amsthm}
\usepackage{epstopdf}
\usepackage[utf8]{inputenc}
\usepackage[usenames,dvipsnames]{color}
\usepackage[table]{xcolor}
\usepackage{hyperref}
\DeclareGraphicsRule{.tif}{png}{.png}{`convert #1 `dirname #1`/`basename #1 .tif`.png}

\theoremstyle{definition}
\newtheorem{example}{Example}

\newenvironment{textblock}{%
   \setlength{\parindent}{0pt}
   \large
   
}{}

\newcommand{\projectname}{Smart Shopping List}
\newcommand{\productname}{Smart Shopping List}
\newcommand{\projectleader}{A. Walliser}
\newcommand{\documentstatus}{In process}
%\newcommand{\documentstatus}{Submitted}
%\newcommand{\documentstatus}{Released}
\newcommand{\version}{V. 1.0}

\begin{document}
\begin{titlepage}
\begin{flushright}
\includegraphics[scale=.5]{htlleondinglogo.png}\\
\end{flushright}

\vspace{10em}

\begin{center}
{\Huge Project Proposal} \\[3em]
{\LARGE \productname} \\[3em]
\end{center}

\begin{flushleft}
\begin{tabular}{|l|l|}
\hline
Project Name & \projectname \\ \hline
Project Leader & \projectleader \\ \hline
Document state & \documentstatus \\ \hline
Version & \version \\ \hline
\end{tabular}
\end{flushleft}

\end{titlepage}
\section*{Revisions}
\begin{tabular}{|l|l|l|}
\hline
\cellcolor[gray]{0.5}\textcolor{white}{Date} & \cellcolor[gray]{0.5}\textcolor{white}{Author} & \cellcolor[gray]{0.5}\textcolor{white}{Change} \\ \hline
October 19, 2018&C. Wagner/A. Walliser&First version \\ \hline
\end{tabular}
\pagebreak

\tableofcontents
\pagebreak

\section{Introduction}

\begin{textblock}
The smart shopping list is an easy to use app which makes shopping and finding recipes to cook easier because you can merge all shopping lists and recipes of your household and have them always with you. In the app you can create and join groups, those groups have shared shopping lists and recipes. \\
You can also find other peoples recipes or get inspired by them. When you select a recipe all needed ingredients will be automatically added to your shopping list unless you mark them as already present. 
\end{textblock}
\pagebreak

\section{Initial Situation}

\begin{textblock}
A typical household must go shopping for groceries at least once a week. A lot of households use grocery list to organise that process. Problems that could occur are that the grocery list gets lost or if the list is in use nobody else can add shopping items to the list. It also could happen that multiple lists get written because of miscommunication between the members of a household. \\
Another problem is if someone wants to cook a dish using a recipe because adding the items to the grocery list manually can be a chore.\\
Those two processes could be simplified by using a grocery list app. Apps like this already exist but often they can only be used by one person at a time, have no or bad option for recipe management or they have a discursive user-interface.
\end{textblock}

\pagebreak

\section{General Conditions and Constraints}

\begin{textblock}
The proposed system has to deal with the following constraints: \\\\
Framework conditions:
\begin{itemize}
\item needed know-how: Java, databases, mobile-development
\item Project release: 13.6.2019
\item Development environment: IntelliJ \\\\
\end{itemize}

Technical conditions:
\begin{itemize}
\item The GUI must be intuitive
\item An external database needs to be setup
\item Possibly cross-platform development
\item The database must always be avaiable
\end{itemize}
\end{textblock}

\pagebreak

\section{Project Objectives and System Concepts}

\begin{textblock}
The project objectives can be summarized as follows:
\begin{itemize}
\item Users can create/enter groups with shared grocery list.
\item Every group member can add/remove items from the grocery list.
\item A user can enter multiple groups.
\item Recipes show the needed ingredients which are automatically added to the grocery list.
\item Users can share recipes in their groups or with the world.
\item The items in the grocery list are sorted by categories which can be altered by the user.
\end{itemize}
\end{textblock}

\pagebreak

\section{Opportunities and Risks}

\begin{textblock}
Potential customers:
\begin{itemize}
\item Every person that needs to go shopping for groceries is a potential costumer.\\
Especially household with more than one member.\\\\
\end{itemize}

The project has the following opportunities:
\begin{itemize}
\item The processes of shopping gets simplified.
\item Decreases the time of finding recipes.
\item The time needed for shopping will drastically decrease.\\\\
\end{itemize}

The following risk have to be taken into account.
\begin{itemize}
\item People don’t use the app because they don’t want to readjust.
\item People use the app of competitors rather than ours.
\end{itemize}
\end{textblock}

\pagebreak

\section{Planning}

\begin{textblock}
\begin{itemize}
\item Project end: 13.6.2019
\item Project start: 23.10.2018
\item First prototype: 25. 2. 2019 \\\\
\end{itemize}

Milestones:
\begin{itemize}
\item Setup basic Database: 13.12.2018
\item Login: 14.1.2019
\item basic Graphical-User-Interface: 25. 2. 2019
\item Implementation of a shopping list: 25. 2. 2019
\item Implementation of the recipe book: 4.4.2019 \\\\
\end{itemize}


Project leader: Alexander Walliser \\
Lead programmer: Clements Wagner

\end{textblock}
\end{document}  