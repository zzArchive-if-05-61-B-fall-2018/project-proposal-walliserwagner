\documentclass[12pt]{article}
\usepackage{geometry}                % See geometry.pdf to learn the layout options. There are lots.
\geometry{letterpaper}                   % ... or a4paper or a5paper or ... 
\usepackage{graphicx}
\usepackage{amssymb}
\usepackage{amsthm}
\usepackage{epstopdf}
\usepackage[utf8]{inputenc}
\usepackage[usenames,dvipsnames]{color}
\usepackage[table]{xcolor}
\usepackage{hyperref}
\DeclareGraphicsRule{.tif}{png}{.png}{`convert #1 `dirname #1`/`basename #1 .tif`.png}

\theoremstyle{definition}
\newtheorem{example}{Example}

\newcommand{\projectname}{Smart Shopping List}
\newcommand{\productname}{Smart Shopping List}
\newcommand{\projectleader}{A. Walliser}
\newcommand{\documentstatus}{In process}
%\newcommand{\documentstatus}{Submitted}
%\newcommand{\documentstatus}{Released}
\newcommand{\version}{V. 1.0}

\begin{document}
\begin{titlepage}
\begin{flushright}
\includegraphics[scale=.5]{htlleondinglogo.png}\\
\end{flushright}

\vspace{10em}

\begin{center}
{\Huge Project Proposal} \\[3em]
{\LARGE \productname} \\[3em]
\end{center}

\begin{flushleft}
\begin{tabular}{|l|l|}
\hline
Project Name & \projectname \\ \hline
Project Leader & \projectleader \\ \hline
Document state & \documentstatus \\ \hline
Version & \version \\ \hline
\end{tabular}
\end{flushleft}

\end{titlepage}
\section*{Revisions}
\begin{tabular}{|l|l|l|}
\hline
\cellcolor[gray]{0.5}\textcolor{white}{Date} & \cellcolor[gray]{0.5}\textcolor{white}{Author} & \cellcolor[gray]{0.5}\textcolor{white}{Change} \\ \hline
October 19, 2018&C. Wagner/A. Walliser&First version \\ \hline
November 18, 2018&C. Wagner/A. Walliser&Second version \\ \hline
\end{tabular}
\pagebreak

\tableofcontents
\pagebreak

\section{Introduction}

The smart shopping list is an easy to use app which makes shopping and finding recipes to cook easier because all shopping lists and recipes of a household can be shared. In the app the user can create and join groups, those groups have shared shopping lists and recipes.

One can also find other people's recipes or get inspired by them. When a recipe is selected all needed ingredients will be automatically added to the shopping list unless they are marked as already present. 

\pagebreak

\section{Initial Situation}

\subsection{Overview}

Members of a typical household must go shopping for groceries at least once a week. A lot of households use grocery lists to organise that process. Problems that could occur are that the grocery list gets lost or if the list is in use nobody else can add shopping items to the list. It also could happen that multiple lists get written because of miscommunication between the members of a household.

Furthermore things get more complex when the combination of recipe books and grocery lists is considered. The items found in different recipe books have to be manually transferred to the shopping list. 

These processes could be simplified by using a grocery list app. Apps like this already exist. In the following the most popular apps in this area are listed and their pros and cons are evaluated.

\subsection{Examples}
\subsubsection{Shopping List}
The app shopping list is very simple designed, so the user has only the ability to create a shopping list where they can add and remove items. There are no features like shared lists or recipe books. The items on the list can not be categorized, and when adding an item, there is no quick select feature so the user always has to write the whole item name. When clicking on an item on the list it can be marked as bought, edited or deleted, but there is no feature to delete all items that are marked as bought. So after marking all items as bought, every single item has to be manually deleted. 

\subsubsection{Bring}
Bring is a multi-user shopping list app where the user can share shopping lists in a group. The app provides the user with default items those have icons matching the product. When an item is added to a shopping list the amount can be stated. The app provides a recipe book but the recipes added by the user are mixed with recipe suggestions which makes it harder to find recipes. New items can be added by the user. They can choose a name and one of the already existing icons for their item. Shopping items are categorized which make them easier to find but the categories can not be altered by the user in any way. That makes it hard to choose which category a new item should be in because the category where it actually belongs is not provided. Categories can also get redundant if the user is in no need for a certain type of item for example vegetarians do not need a "Meat and Fish" category. Bring has a web-interface   and is available for both android and IOS.  

\subsubsection{Die Einkaufliste}
In the app Die Einkaufsliste the user has the ability to share shopping lists via a link that can be send by email. The items on the list are categorized, but the order of the categories can not be changed. There also is a quick selection feature when adding an item and to remove an item the user only has to tick it and click on a button to remove all ticked items. There is no feature for creating groups or managing recipes.

\pagebreak

\section{General Conditions and Constraints}

\subsection{Conditions}
It is very essential for our app to have an user-interface that is easy to use because every person that goes shopping is a potential customer. Some of them are quickly overwhelmed when using apps therefore we want to design our app similar to the UI goolgle uses, that for example  is used in  gmail, goolgle drive or google docs. We would benefit from using an UI that is designed like the UI google uses because most android users are familiar with it, so it is much easier for them to get used to our app.

In order to allow the user to share their lists and recipes, we have to setup a database that is always reachable, so we can guarantee that the user can update the shared lists and recipes at any time.

\subsection{Constraints}
Users can only update the shared lists and recipes if they have an internet connection. Since the app is mostly used while shopping where mobile data is used, the app should use as little data volume as possible. Because some users do not have mobile data or already consumed it, we have to provide an off-line feature where the user can access the lists. We want to achieve this by saving the lists local and synchronise the lists when an internet connection is available.
Due to limited time we are not going to provide cross-platform,so the user needs an android device to use our app.
\subsection{Technical Conditions}
Since we want to create an android app and plan to use Android Studio, we have to get familiar with it. As programming language we want to use Java, because we already worked with it, but on the other hand we have no experience with android development, so we have to learn it from scratch and extend our current knowledge of Java.
\pagebreak

\section{Project Objectives and System Concepts}
A user can have multiple shopping lists, where they can add and remove items. The Items are categorized, these categories can be altered, however there is a general category that can not be changed.This category obtains all items that are not assigned to an other category. When adding an item to the list the user can look for them in a search bar or select them via their category and when creating a new item the user can deicide which category it belongs to. Items can be marked as bought, all bought items will be moved to the end of the list and when pressing a remove button all marked items will be removed at once. The user can also remove items manually.

Ever user also has their own recipe book where they can create recipes or add recipes from groups or recipes that are posted by an other user. The recipes can also be categorized. A recipe contains a description, the necessary items, optional items and optionally a picture of the meal. The ingredients of a recipe can be added to the shopping list automatically.

All the recipes, categories, items and lists are saved on a database, so the user can access them from different devices and also local, so the user can use the app without internet connection.

Users should be able to create and enter groups with shared grocery lists. A group has its own lists, categories and recipes. When creating a group the default categories, and items will be the same as the creators. These categories and items can be altered by every member of the group. A group member can add their own recipes to the groups recipe book.

The recipes, categories, items and lists of a group are also saved on a database. The last downloaded version of a groups lists can be accessed offline, however the user can only mark items as bought, but can not remove nor add items. When the user obtains internet connection, the lists will be updated and the items that are marked as bought will also be marked in the groups lists.

\pagebreak

\section{Opportunities and Risks}
\subsection{Potential Customers}
Every person that needs to go shopping for groceries is a potential costumer.
Especially households with more than one member.

\subsection{Opportunities}
One of the biggest opportunities is that the process of shopping gets simplified, because the members of a household do not have to merge their lists any more and when using the app it is less likely to forget some items. An other advantage would be that the time for finding new or even the own recipes is drastically decreased.

\subsection{Risks}
One of the risks we have to take into account is than people use the app of competitors rather than ours because they are well-known or the user prefers them. An other risk could be that do not want to use a shopping list app because they do not want to readjust.

\pagebreak

\section{Planning}
\subsection{Overview}
At first we have to set-up a basic database where we can store user-information like e-mail, user-name, password for the login, the database will be extended later on. The next step is to implement a login. Next up we have to work on the graphical-user-interface until it got to a point where we can easily start to implement the shopping list. When the shopping list is finished we start implementing the recipe book. After completing each of these we test them thoroughly. After finishing these steps we want to concentrate on finding and fixing bugs. If there is time left we will add some more features.
\begin{itemize}
\item Project start: 23.10.2018
\item Project end: 13.6.2019
\end{itemize}

\subsection{Milestones}
\begin{tabular}{|l|l|}
\hline
\cellcolor[gray]{0.5}\textcolor{white}{Date} &
\cellcolor[gray]{0.5}\textcolor{white}{Milestone} \\ \hline
December 13, 2018 & Database set-up \\ \hline
January 14, 2019 & Login finished \\ \hline
February 25, 2019 & Prototype Graphical-User-Interface finished \\ \hline
March 25, 2019 & shopping list finished \\ \hline
April 4, 2019 & recipe book finished \\ \hline
\end{tabular}

\subsection{Team}
Project leader: Alexander Walliser \\
Programmer: Clements Wagner
\subsection{Recources}
\begin{itemize}
\item Server
\item IntelliJ licence
\item Android devices for testing
\end{itemize}
\end{document}  