\documentclass[12pt]{article}
\usepackage{geometry}                % See geometry.pdf to learn the layout options. There are lots.
\geometry{letterpaper}                   % ... or a4paper or a5paper or ... 
\usepackage{graphicx}
\usepackage{amssymb}
\usepackage{amsthm}
\usepackage{epstopdf}
\usepackage[utf8]{inputenc}
\usepackage[usenames,dvipsnames]{color}
\usepackage[table]{xcolor}
\usepackage{hyperref}
\DeclareGraphicsRule{.tif}{png}{.png}{`convert #1 `dirname #1`/`basename #1 .tif`.png}

\theoremstyle{definition}
\newtheorem{example}{Example}

\newcommand{\projectname}{Smart Shopping List}
\newcommand{\productname}{Smart Shopping List}
\newcommand{\projectleader}{A. Walliser}
\newcommand{\documentstatus}{In process}
%\newcommand{\documentstatus}{Submitted}
%\newcommand{\documentstatus}{Released}
\newcommand{\version}{V. 1.0}

\begin{document}
\begin{titlepage}
\begin{flushright}
\includegraphics[scale=.5]{htlleondinglogo.png}\\
\end{flushright}

\vspace{10em}

\begin{center}
{\Huge Project Proposal} \\[3em]
{\LARGE \productname} \\[3em]
\end{center}

\begin{flushleft}
\begin{tabular}{|l|l|}
\hline
Project Name & \projectname \\ \hline
Project Leader & \projectleader \\ \hline
Document state & \documentstatus \\ \hline
Version & \version \\ \hline
\end{tabular}
\end{flushleft}

\end{titlepage}
\section*{Revisions}
\begin{tabular}{|l|l|l|}
\hline
\cellcolor[gray]{0.5}\textcolor{white}{Date} & \cellcolor[gray]{0.5}\textcolor{white}{Author} & \cellcolor[gray]{0.5}\textcolor{white}{Change} \\ \hline
October 19, 2018&C. Wagner/A. Walliser&First version \\ \hline
\end{tabular}
\pagebreak

\tableofcontents
\pagebreak

\section{Introduction}

The smart shopping list is an easy to use app which makes shopping and finding recipes to cook easier because all shopping lists and recipes of a household can be shared. In the app the user can create and join groups, those groups have shared shopping lists and recipes.

One can also find other people's recipes or get inspired by them. When a recipe is selected all needed ingredients will be automatically added to the shopping list unless they are marked as already present. 

\pagebreak

\section{Initial Situation}

\subsection{Overview}

Members of a typical household must go shopping for groceries at least once a week. A lot of households use grocery lists to organise that process. Problems that could occur are that the grocery list gets lost or if the list is in use nobody else can add shopping items to the list. It also could happen that multiple lists get written because of miscommunication between the members of a household.

Furthermore things get more complex when the combination of recipe books and grocery lists is considered. The items found in different recipe books have to be manually transferred to the shopping list. 

These processes could be simplified by using a grocery list app. Apps like this already exist. In the following the most popular apps in this area are listed and their pros and cons are evaluated.

\subsection{Examples}
\subsubsection{Listonic}
What is the main focus of this app?

Pros:

\begin{itemize}
\item Well structured.
\end{itemize}
Cons:
\begin{itemize}
\item No option for recipe management.
\item There is no possibility to share shopping lists.
\end{itemize}

\subsubsection{Bring}
Pros:
\begin{itemize}
\item Shopping items are categorised.
\item There is a recipe option.\\
\end{itemize}
Cons:
\begin{itemize}
\item Categories can not be changed or extended.
\item Own recipe book and other people's recipes are not separated.
\item Recipes can not be shared in a group.
\end{itemize}
Die Einkaufsliste:\\
Pros:
\begin{itemize}
\item Lists can be shared via link.
\end{itemize}
Cons:
\begin{itemize}
\item No option for recipe management.
\end{itemize}

\pagebreak

\section{General Conditions and Constraints}

\subsection{Conditions}
It is very essential for our app to have an user-interface that is easy to use because every person that goes shopping is a potential customer. Some of them are quickly overwhelmed when using apps therefore we want to design our app similar to youtubes UI.
%Conditions:
%\begin{itemize}
%\item A shopping list should be usable by anyone therefore the GUI should also be understandable for non tech-savvy people.
%\item We need a database that is always reachable.
%\end{itemize}
%Constraints:
%\begin{itemize}
%\item The user needs a internet connection.
%\item The app will often be used while shopping where mobile data is used therefore the app should not use to much data volume. 
%\item The user needs an android device to use the app.
%\end{itemize}

\pagebreak

\section{Project Objectives and System Concepts}

Users should be able to create and enter groups with shared grocery lists. In those groups members can add and remove items from the grocery list. The user should be able to enter multiple groups. The grocery list can be sorted by categories. Default categories will be provided those can be altered or extended by the user. The user can add items to categories so they can be found more easily later on. The app should also help with recipe management, recipes can be saved in the recipe book. The ingredients of the recipe will be added automatically to the grocery list when selected. Recipes can be shared in a group or can be posted for other users to find. 

\pagebreak

\section{Opportunities and Risks}

Potential customers:
\begin{itemize}
\item Every person that needs to go shopping for groceries is a potential costumer.\\
Especially households with more than one member.\\\\
\end{itemize}

The project has the following opportunities:
\begin{itemize}
\item The processes of shopping gets simplified.
\item Decreases the time of finding recipes.
\item The time needed for shopping will drastically decrease.\\\\
\end{itemize}

The following risk have to be taken into account.
\begin{itemize}
\item People don not use the app because they don not want to readjust.
\item People use the app of competitors rather than ours.
\end{itemize}

\pagebreak

\section{Planning}

\begin{itemize}
\item Project end: 13.6.2019
\item Project start: 23.10.2018 \\\\
\end{itemize}

Milestones:
\begin{itemize}
\item Database : 13.12.2018
\item Login finished: 14.1.2019
\item basic Graphical-User-Interface finished: 25. 2. 2019
\item shopping list finished: 25. 2. 2019
\item recipe book finished: 4.4.2019 \\\\
\end{itemize}


Project leader: Alexander Walliser \\
Lead programmer: Clements Wagner

\end{document}  